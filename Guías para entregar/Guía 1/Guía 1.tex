\input{Header.tex}

\begin{document}

%%%%%%%%%%%%%%%%%%%%%%%%%
%		Caratula		%
%%%%%%%%%%%%%%%%%%%%%%%%%

\input{Caratula.tex}


%%%%%%%%%%%%%%%%%%%%%
%		Informe		%
%%%%%%%%%%%%%%%%%%%%%

\section*{Ejercicio 1}
\begin{itemize}
	\item[d)] $\mathbf{R \left[ x \left( nT \right) \right] = 5nT x^2 \left( nT \right)}$ 
	
		\textbf{Invariancia:}
		
		$R \left[ x \left( nT - T \right) \right] = 5nT x^2 \left( nT -T \right)$ 
		
		$T \left[ R \left[ x \left( nT \right) \right] \right] = T \left[ 5nT x^2 \left( nT \right) \right] = 5nT x^2 \left( nT -T \right)$ 
		
		Es tiempo invariante.
		
		\textbf{Causalidad:} Es causal ya que no depende de entradas futuras.
		
		\textbf{Linealidad:}
		
		 $R \left[ ax_1 \left( nT \right) + bx_2 \left( nT \right) \right] = a5nT {x_{1}}^{2} \left( nT -T \right) + b5nT {x_{2}}^{2} \left( nT -T \right) = aR \left[ x_1 \left( nT \right) \right] + bR \left[ x_2 \left( nT \right) \right]$

		Es un sistema lineal.
		
	\item[e)] $\mathbf{R \left[ x \left( nT \right) \right] = 3x \left( nT - 3T\right)}$ 
	
		\textbf{Invariancia:}
		
		$R \left[ x \left( nT - T \right) \right] = 3x \left( nT - 3T -T \right)$ 
		
		$T \left[ R \left[ x \left( nT \right) \right] \right] = T \left[ 3x \left( nT - 3T \right) \right] = 3x \left( nT - 3T -T \right)$ 
		
		Es tiempo invariante.
		
		\textbf{Causalidad:} Es causal ya que no depende de entradas futuras.
		
		\textbf{Linealidad:}
		
		 $R \left[ ax_1 \left( nT \right) + bx_2 \left( nT \right) \right] = a3 x_{1} \left( nT - 3T \right) + b3 x_{2} \left( nT - 3T \right) = aR \left[ x_1 \left( nT \right) \right] + bR \left[ x_2 \left( nT \right) \right]$

		Es un sistema lineal.
		
	\item[i)] $\mathbf{R \left[ x \left( nT \right) \right] = x \left( nT + T\right) e^{-nT}}$ 
	
		\textbf{Invariancia:}
		
		$R \left[ x \left( nT - T \right) \right] =  x \left( nT \right) e^{-nT + T}$ 
		
		$T \left[ R \left[ x \left( nT \right) \right] \right] = T \left[  x \left( nT \right) e^{-nT + T} \right] = x \left( nT \right) e^{-nT + T}$ 
		
		Es tiempo invariante.
		
		\textbf{Causalidad:} No es causal ya que depende de entradas futuras.
		
		\textbf{Linealidad:}
		
		 $R \left[ ax_1 \left( nT \right) + bx_2 \left( nT \right) \right] = a x_{1} \left( nT \right) e^{-nT + T} + b x_{2} \left( nT \right) e^{-nT + T} = aR \left[ x_1 \left( nT \right) \right] + bR \left[ x_2 \left( nT \right) \right]$

		Es un sistema lineal.		

\end{itemize}

\section*{Ejercicio 2b}
La ecuación en diferencia del sistema se vale de la función auxiliar $e(nT)$.

\begin{enumerate}
	\item	$e(nT) = x(nT) + e(nT - T) - 0.5e(nT - 2T)$
	\item	$y(nT) = e(nT) + e(nT - T)$
\end{enumerate}

\section*{Ejercicio 9}
La ecuación en diferencia del sistema es
\begin{equation*}
	y(nT) = 0.5x(nT - 2T) + \alpha y(nT - T) + \beta y(nT - 2T)
\end{equation*}
se obtienen los siguientes resultados.

\begin{figure}[H]
\centering
\begin{subfigure}{.7\textwidth}
\centering
	\includegraphics[width=\textwidth]{Imagenes/9-impulso.png}
\end{subfigure}
\begin{subfigure}{.7\textwidth}
\centering
	\includegraphics[width=\textwidth]{Imagenes/9-escalon.png}
\end{subfigure}
\end{figure}

\begin{center}
	\textcolor{red}{\textbf{Para estimar la respuesta el impulso en el caso de a...}}
\end{center}



\end{document}