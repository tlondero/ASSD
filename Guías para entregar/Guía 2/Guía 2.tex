\documentclass[a4paper]{article}
\usepackage[utf8]{inputenc}
\usepackage[spanish, es-tabla, es-noshorthands]{babel}
\usepackage[table,xcdraw]{xcolor}
\usepackage[a4paper, footnotesep = 1cm, width=20cm, top=2.5cm, height=25cm, textwidth=18cm, textheight=25cm]{geometry}
%\geometry{showframe}

\usepackage{tikz}
\usepackage{amsmath}
\usepackage{amsfonts}
\usepackage{amssymb}
\usepackage{float}
\usepackage{graphicx}
\usepackage{caption}
\usepackage{subcaption}
\usepackage{multicol}
\usepackage{multirow}
\setlength{\doublerulesep}{\arrayrulewidth}
\usepackage{booktabs}
\usepackage{mathrsfs,amsmath}
\usepackage{hyperref}
\hypersetup{
    colorlinks=true,
    linkcolor=blue,
    filecolor=magenta,      
    urlcolor=blue,
    citecolor=blue,    
}

\newcommand{\quotes}[1]{``#1''}
\usepackage{array}
\newcolumntype{C}[1]{>{\centering\let\newline\\\arraybackslash\hspace{0pt}}m{#1}}
\usepackage[american]{circuitikz}
\usetikzlibrary{calc}
\usepackage{fancyhdr}
\usepackage{units} 

\graphicspath{./Imagenes}

\pagestyle{fancy}
\fancyhf{}
\lhead{22.05 ASSD}
\rhead{Mechoulam, Lambertucci, Rodriguez, Londero}
\rfoot{Página \thepage}
\usepackage{amsmath}
\begin{document}

%%%%%%%%%%%%%%%%%%%%%%%%%
%		Caratula		%
%%%%%%%%%%%%%%%%%%%%%%%%%

\begin{titlepage}
\newcommand{\HRule}{\rule{\linewidth}{0.5mm}}
\center
\mbox{\textsc{\LARGE \bfseries {Instituto Tecnológico de Buenos Aires}}}\\[1.5cm]
\textsc{\Large 22.05 Análisis de Señales y Sistemas Digitales}\\[0.5cm]


\HRule \\[0.6cm]
{ \Huge \bfseries Trabajo práctico N$^{\circ}$2}\\[0.4cm] 
\HRule \\[1.5cm]


{\large

\emph{Grupo 3}\\
\vspace{3px}

\begin{tabular}{lr} 	
\textsc{Mechoulam}, Alan  &  58438\\
\textsc{Lambertucci}, Guido Enrique  & 58009 \\
\textsc{Rodriguez Turco}, Martín Sebastian  & 56629 \\
\textsc{Londero Bonaparte}, Tomás Guillermo  & 58150 \\
\end{tabular}

\vspace{20px}

\emph{Profesores}\\
Jacoby, Daniel Andres\\
Belaustegui Goitia, Carlos F.\\
Iribarren, Rodrigo Iñaki\\
\vspace{3px}
%\textsc{} \\	

\vspace{100px}

\begin{tabular}{ll}

Presentado: & 15/05/20\\

\end{tabular}

}

\vfill

\end{titlepage}



%%%%%%%%%%%%%%%%%%%%%
%		Informe		%
%%%%%%%%%%%%%%%%%%%%%

\section*{Ejercicio 1}
\begin{itemize}
	\item[2b)] 
	Se habia llegado al resultado:
	\begin{align}
 \begin{cases} 
      x(nT) = e(nT) -  e(nT - T)+0.5e(nT - 2T)    \\
      y(nT) = e(nT) + e(nT - T) 
   \end{cases} 
   \begin{Huge}
   \xrightarrow{\mathcal{Z}}
   \end{Huge} 
    \begin{cases} 
      X(z) = E(z) \cdot (1-  z^{-1}+ 0.5 \cdot  z^{-2}) \\
      Y(z) = E(z)\cdot (1 +  z^{-1})
   \end{cases}
	\end{align}
Igualando las expresiones:
\begin{align}
Y(z) \cdot (1-  z^{-1}+ 0.5 \cdot  z^{-2}) = X(z) \cdot (1 +  z^{-1})\xrightarrow{\mathcal{Z}^{-1}}
y(n) = x(n) + x(n-1)+y(n-1) -0.5 \cdot y(n-2) 
\end{align}
\item[9)] 	Se habia llegado al resultado:
	\begin{align}
	y(n)= 0.5 \cdot x(n-2) + \alpha  \cdot y(n-1) + \beta  \cdot y(n-2)    \xrightarrow{\mathcal{Z}} Y(z) = 0.5 \cdot z^{-2} \cdot X(z) + \alpha  \cdot z^{-1} \cdot Y(z) + \beta z^{-2}  \cdot Y(z) 
	\end{align}
	Despejando la transferencia dado que esta es la transformada de la respuesta al impulso.
	\begin{align}
	H(z)=\frac{1}{2}\cdot \frac{1}{z^2-\alpha z - \beta} = \frac{1}{2}\cdot \left( \frac{\frac{1}{z_1-z_2}}{z-z_1} + \frac{\frac{1}{z_2-z_1}}{z-z_2} \right) \ \ \ \ \ \ \ \ \ \ z_{1,2}=\frac{\alpha \pm \sqrt{\alpha ^2 + 4\beta} }{2}
		\end{align}
		\begin{align}
	h(n)=\frac{1}{2}\cdot \frac{1}{z_1-z_2}\cdot (z_1^{n-1}-z_2^{n-1})\cdot u(n-1)
	\end{align}
\end{itemize}	
\section*{Ejercicio 3}
\begin{itemize}
	\item[a)]
		Para analizar la estabilidad de (\ref{eq:hz}) se utilizaron 2 métodos, el primero es verificar que el modulo de las raices del denominador sean menor a 1, y el otro es el metodo de \href{https://en.wikibooks.org/wiki/Control_Systems/Jurys_Test}{Jury-Marden Stability Criterion}
		\begin{equation}
		H(z)=\frac{z^6}{6z^6+5z^5+4z^4+3z^3+2z^2+z+1}
		\label{eq:hz}
		\end{equation}
		Las raices del denominador serán:
		\begin{equation}
		z_{1,2}=-0.703387 \pm j \cdot 0.365055 \ \ \ \ 		z_{3,4}=-0.116036 \pm j\cdot  0.731154 \ \ \ \ z_{5,6}=0.402756 \pm j\cdot  0.567471
		\end{equation}
		los cuales todos cuentan con módulo menor a 1, lo cual indica que el sistema es estable.
		Para el segundo análisis de estabilidad
		primero se comprobaron las hipótesis de Jury, las cuales son:
		Sea un sistema descrito como $H(z) = \frac{N(z)}{D(z)}$
		\\
		\begin{itemize}
		\item D(1) $>$ 0 $\rightarrow D(1)=21$
				\item $(-1)^N \cdot D(-1) > 0 \rightarrow (-1)^6 \cdot D(-1)=4$
				\item $a_0 < a_n \rightarrow 1<6$
		\end{itemize}
		Dado que cumple con todas las hipótesis se procedió a generar la matriz de Jury:
		\begin{table}[H]
\centering
\begin{tabular}{llllllll}
\multicolumn{1}{c}{Rows} & $z^0$ & $z^1$ & $z^2$ & $z^3$ & $z^4$ & $z^5$ & $z^6$ \\ \hline
\multicolumn{1}{l|}{1} & 1 & 1 & 2 & 3 & 4 & 5 & 6 \\
\multicolumn{1}{l|}{2} & 6 & 5 & 4 & 3 & 2 & 1 & 1 \\
\multicolumn{1}{l|}{3} & -35 & -29 & -22 & -15 & -8 & -1 & 0 \\
\multicolumn{1}{l|}{4} & 0 & -1 & -8 & -15 & -22 & -29 & -35 \\
\multicolumn{1}{l|}{5} & 1224 & 1007 & 755 & 503 & 251 & 0 & 0 \\
\multicolumn{1}{l|}{6} & 0 & 0 & 251 & 503 & 755 & 1007 & 1224 \\
\multicolumn{1}{l|}{7} & 1435175 & 1106315 & 734615 & 362915 & 0 & 0 & 0 \\
\multicolumn{1}{l|}{8} & 0 & 0 & 0 & 362915 & 734615 & 1106315 & 1435175 \\
\multicolumn{1}{l|}{9} & 1928019983400 & 1321152827400 & 652802774400 & 0 & 0 & 0 & 0
\end{tabular}
\end{table}

\end{itemize}
De aquí para asegurar la estabilidad basta con verificar que los  $|b_0| > |b_{5}| \ \|c_0| > |c_{4}| \ \ |d_0| > |d_{3}| \ \|e_0| > |e_{2}|$
Los cuales verifican, por lo cual se define que el sistema es estable.


\end{document}