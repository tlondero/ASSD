\input{../Informe/Header.tex}

\begin{document}

\subsection{Conversor Sigma Delta $\Sigma \Delta$}

\subsubsection{Introducción}
El conversor sigma-delta o delta-sigma según la literatura consultada, tuvo su aparición durante los años 60's y 70's. Aunque su nombre pueda sugerir un tipo de tecnología muy compleaja la realidad es que no lo es


\subsubsection{Arquitectura}

A continuación presentamos la topología implementada:

\begin{figure}[H]
	\centering
	\includegraphics[width=0.7\linewidth]{ImagenesEjercicio2/diagramaEnBloques}
	\caption{Modulador $\Sigma\Delta$ de primer orden}
	\label{fig:diagramaenbloques}
\end{figure}

Como se puede observar el modulador $\Sigma\Delta$ se basa en un circuito de realimentación negativa que incorpora un un cuantificador y un DAC en su interior. 



\subsubsection{Modulador}
La salida del modulador $\Sigma\Delta$ es un bit-stream de 1's y 0's que representan los cambios que ocurren en la señal de entrada.



\subsubsection{Demodulador}

\subsubsection{Noise Shaping}

\begin{figure}[H]
	\centering
	\includegraphics[width=0.7\linewidth]{ImagenesEjercicio2/NoiseShappingAN}
	\caption{}
	\label{fig:noiseshappingan}
\end{figure}


\subsubsection{Simulación de aplicación real}
En este ejemplo vamos a simular la digitalización de una conversación, la voz humana... rango audible de ... por lo tanto es 

\subsubsection{Ejemplos varios}

\end{document}