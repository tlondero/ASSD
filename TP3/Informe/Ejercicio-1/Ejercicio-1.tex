\input{../Informe/Header.tex}

\begin{document}

\subsection{Introducción}

\subsection{Máxima frecuencia de entrada sin Sample \& Hold}

De la datasheet del ADC0808, se tiene que si $V_{CC} = V_{REF+} = 5.12V$ y $V_{REF-} = 0V$, la resolución será de $20 \frac{mV}{bit}$. Si se utiliza la frecuencia de clock $f_{CLK}$ típica utilizada en la datasheet de $640kHz$, el tiempo de conversión $t_C$ máximo será de $116\mu s$. Esto implica que la entrada no deberá de tener una pendiente mayor a $\frac{20mV}{116\mu s}$ para no introducir error en la cuantización de la señal.


Si la señal de entrada se encuentra en el peor caso, es decir, con una excursión de tensión de $-0.1V + V_{REF-}$ a $5.12V + 0.1V$; esta se encuentra montada sobre un nivel de continua igual a $(5.22V - (-0.1V))/2 = 2.66V$; y esta se puede considerar senoidal gracias a la teoría desarrollada por Fourier; se tiene que la amplitud pico máxima de la senoidal podrá ser $2.66V$. Luego, asumiendo el peor caso de la pendiente de la senoidal, para un ángulo igual a cero radianes, lo que permite utilizar la aproximación paraxial, se tiene que
\\

\begin{equation}
\left. \frac{d \left( 2.66V \cdot Sin \left( 2\pi f_{in_{max}} t \right) \right)}{dt} \right|_{t=0} = 2.66V \cdot 2\pi f_{in_{max}} = \frac{20mV}{116\mu s}
\end{equation}
\\

Finalmente, se obtiene una frecuencia máxima de $f_{in_{max}} = 10.3Hz$.
\subsection{DAC}
Se utilizó el integrad DAC0800, un conversor D/A de 8 bits con salida diferencial de corriente.
Para convertir esta corriente en un nivel de tensión se utilizó el circuito propuesto por la hoja de datos que se muestra a continuación:
\begin{figure}[H]
	\centering
	\includegraphics[width=0.8\textwidth]{ImagenesEjercicio1/dacout.png}
\caption{Configuración saldia DAC.}
	\label{fig:dacout}
\end{figure}
La salida va de 0 a $V_{fs}= I_{fs}cdot R_L$.\\
En cuanto al pinout del DAC es el siguiente:
\begin{figure}[H]
	\centering
	\includegraphics[width=0.5\textwidth]{ImagenesEjercicio1/dacpinout.png}
\caption{Pinout DAC0800.}
	\label{fig:dapinout}
\end{figure}
Donde los pines desde B1 a B8 son las entradas digitales, siendo B1 el bit mas significativo. En cuanto a la salida, es por corriente, y corresponde a los pines 6 y 8, cabe mencionar que dichas corrientes son complementaria, osea su suma da 0. Los pines 2 y 3 son las tensiones $V^+$ Y $V^-$ fueron conectadas a 5V y a GND respectivamente, mientras que las tensiones de referencia, si bien dice tensiones, uno debe proveer corrientes de referencia, esto se hace utilizando una resistencia entre vcc y $V_{ref^+}$, al igual que una entre GND y $V_{ref^-}$. En cuanto al pin de threshold fue conectado a masa, finalmente en el pin de comp se conectó un capacitor de 100nF.
\end{document}