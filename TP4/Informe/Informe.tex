\input{Header.tex}

\begin{document}

%%%%%%%%%%%%%%%%%%%%%%%%%
%		Caratula		%
%%%%%%%%%%%%%%%%%%%%%%%%%

\input{Caratula.tex}

%%%%%%%%%%%%%%%%%%%%%
%		Indice		%
%%%%%%%%%%%%%%%%%%%%%

%\tableofcontents
%\newpage

%%%%%%%%%%%%%%%%%%%%%
%		Informe		%
%%%%%%%%%%%%%%%%%%%%%
\begin{center}
	\Huge{\textcolor{red}{\textbf{HAY QUE HACER EL INFORME.}}}
\end{center}

\textbf{\textit{Resumen: Describe en pocos renglones todo el trabajo, haciendo hincapié en los ensayos y los resultados.}}

\section{Introducción}
	Debe haber suficiente material para que un profesional que no conoce el tema para nada, pueda entenderlo. Referenciar libros y tutorial papers que profundicen.
	
\section{Investigación}
\begin{itemize}
	\item Descripción de las líneas de investigación (con referencias).
	\item Descripción de los conceptos más importantes de cada una.
	\item Análisis propio de lo presentado.
	\item Simulaciones de lo más relevante (códigos como apéndice)
	\item Elección del camino y justificación.
\end{itemize}

\section{Aportes}
\begin{itemize}
	\item Descripción y análisis de lo original producido por el grupo.
	\item Simulaciones que justifiquen las ideas, y que prueben su originalidad.
	\item Análisis de resultados
\end{itemize}

\section{Desarrollo}
\begin{itemize}
	\item Viabilidad, caminos alternativos.
	\item Proceso de implementación
	\item Documentación de los resultados: Resumen de lo más relevante, demos y programas van al apéndice.
	\item Evaluación y conclusiones del desarrollo.
\end{itemize}

\section{Referencias}

\end{document}