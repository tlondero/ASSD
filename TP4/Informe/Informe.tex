\documentclass[a4paper]{article}
\usepackage[utf8]{inputenc}
\usepackage[spanish, es-tabla, es-noshorthands]{babel}
\usepackage[table,xcdraw]{xcolor}
\usepackage[a4paper, footnotesep = 1cm, width=20cm, top=2.5cm, height=25cm, textwidth=18cm, textheight=25cm]{geometry}
%\geometry{showframe}

\usepackage{tikz}
\usepackage{amsmath}
\usepackage{amsfonts}
\usepackage{amssymb}
\usepackage{float}
\usepackage{graphicx}
\usepackage{caption}
\usepackage{subcaption}
\usepackage{multicol}
\usepackage{multirow}
\setlength{\doublerulesep}{\arrayrulewidth}
\usepackage{booktabs}
\usepackage{mathrsfs,amsmath}
\usepackage{hyperref}
\hypersetup{
    colorlinks=true,
    linkcolor=blue,
    filecolor=magenta,      
    urlcolor=blue,
    citecolor=blue,    
}

\newcommand{\quotes}[1]{``#1''}
\usepackage{array}
\newcolumntype{C}[1]{>{\centering\let\newline\\\arraybackslash\hspace{0pt}}m{#1}}
\usepackage[american]{circuitikz}
\usetikzlibrary{calc}
\usepackage{fancyhdr}
\usepackage{units} 

\graphicspath{./Imagenes}

\pagestyle{fancy}
\fancyhf{}
\lhead{22.05 ASSD}
\rhead{Mechoulam, Lambertucci, Rodriguez, Londero}
\rfoot{Página \thepage}

\begin{document}

%%%%%%%%%%%%%%%%%%%%%%%%%
%		Caratula		%
%%%%%%%%%%%%%%%%%%%%%%%%%

\begin{titlepage}
\newcommand{\HRule}{\rule{\linewidth}{0.5mm}}
\center
\mbox{\textsc{\LARGE \bfseries {Instituto Tecnológico de Buenos Aires}}}\\[1.5cm]
\textsc{\Large 22.05 Análisis de Señales y Sistemas Digitales}\\[0.5cm]


\HRule \\[0.6cm]
{ \Huge \bfseries Trabajo práctico N$^{\circ}$2}\\[0.4cm] 
\HRule \\[1.5cm]


{\large

\emph{Grupo 3}\\
\vspace{3px}

\begin{tabular}{lr} 	
\textsc{Mechoulam}, Alan  &  58438\\
\textsc{Lambertucci}, Guido Enrique  & 58009 \\
\textsc{Rodriguez Turco}, Martín Sebastian  & 56629 \\
\textsc{Londero Bonaparte}, Tomás Guillermo  & 58150 \\
\end{tabular}

\vspace{20px}

\emph{Profesores}\\
Jacoby, Daniel Andres\\
Belaustegui Goitia, Carlos F.\\
Iribarren, Rodrigo Iñaki\\
\vspace{3px}
%\textsc{} \\	

\vspace{100px}

\begin{tabular}{ll}

Presentado: & 15/05/20\\

\end{tabular}

}

\vfill

\end{titlepage}


%%%%%%%%%%%%%%%%%%%%%
%		Indice		%
%%%%%%%%%%%%%%%%%%%%%

%\tableofcontents
%\newpage

%%%%%%%%%%%%%%%%%%%%%
%		Informe		%
%%%%%%%%%%%%%%%%%%%%%
\begin{center}
	\Large{\textcolor{red}{\textbf{EN ROJO PONGO LO QUE HAY QUE HACER. NO BORRARLO HASTA NO TERMINARLO. RESPETAR FORMATOS.}}}
\end{center}

\textbf{\textit{En el siguiente trabajo se presenta el estudio, investigación y análisis de un proceso de seguimiento del movimiento de un objeto mediante una cámara, siendo conocida su posición inicial.}}

\textcolor{red}{\textbf{\textit{Resumen: falta mencionar ensayos y resultados.}}}

\section{Introducción}
	Una imagen puede ser interpretada como una función bidimenional $f\left( x, y\right)$, donde tanto $x$ como $y$ representan en un plano el epacio visualizado, mientras que la misma función $f\left( x, y\right)$ es la instensidad de la imagen bajo un punto dado. Cuando $x$, $y$ y $f\left( x, y\right)$ son valores cuantizados y discretizados, la imagen se transforma en una imagen digital~\cite{ref:intro1}.
	 	
	El procesamiento de dichas se define como el conjunto de técnicas aplicadas a estas imágenes, con el objetivo extraer información de ellas. Esas actividads cubren un campo que abarca un sin fin de aplicaciones, ya que en estas se vale de maquinaris las maquinas capaces de detectar el la totalidad del espectro electromagnético. Esto significa que se pueden generar imágenes generadas por fuentes que para los humanos no se asocian con imágenes propiamente dichas, como lo son las ondas de radio, entre tantas otras.
	
	Es posible considerar tres tipos de procesos computarizados en el procesamiento de imagenes basados en el nivel de tratamiento que se aplique, siendo así clasificados en bajo, medio y alto nivel. Los primeros incluyen activdades tales como reducción de ruido y aumento de contraste, tareas caracterizadas por el hecho de que tanto la entrada como la salida son imagenes. Las actividades de medio nivel de procesamiento incluyen trabajos de segmentación, es decir, identificar regiones u objetos dentro de las imagenes, descripción y clasificación de dichos. Es así que esta categoría es destacada por sus salidas, ya que suelen ser información extraida de las imagenes a la entrada. Por último, los procesos de alto nivel se caracterizan por no solo reconocer objetos y analizarlos, sino tambien por darles un tratado normalmente asociado con la visión, tales así como ``darles sentido''.
	
	De esta forma, en este trabajo se centra en procesos de medio nivel.
		
	Se define un pixel como el mínimo elemento que compone un a imagen digital.

	\textcolor{red}{Debe haber suficiente material para que un profesional que no conoce el tema para nada, pueda entenderlo. Referenciar libros y tutorial papers que profundicen.}
	
\section{Investigación}
\textcolor{red}{
\begin{itemize}
	\item Descripción de las líneas de investigación (con referencias).
	\item Descripción de los conceptos más importantes de cada una.
	\item Análisis propio de lo presentado.
	\item Simulaciones de lo más relevante (códigos como apéndice)
	\item Elección del camino y justificación.
\end{itemize}
}

\section{Aportes}
\textcolor{red}{
\begin{itemize}
	\item Descripción y análisis de lo original producido por el grupo.
	\item Simulaciones que justifiquen las ideas, y que prueben su originalidad.
	\item Análisis de resultados
\end{itemize}
}

\section{Desarrollo}
\textcolor{red}{
\begin{itemize}
	\item Viabilidad, caminos alternativos.
	\item Proceso de implementación
	\item Documentación de los resultados: Resumen de lo más relevante, demos y programas van al apéndice.
	\item Evaluación y conclusiones del desarrollo.
\end{itemize}
}

\begin{thebibliography}{9}
\bibitem{ref:intro1}
Rafael C. Gonzalez, Richar E. Woods and Steven L. Eddins. \textit{Digital Image Processing Using MATLAB}. Prentice Hall, 2nd ed, 2002.
\end{thebibliography}

\end{document}