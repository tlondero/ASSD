\input{Header.tex}

\begin{document}

%%%%%%%%%%%%%%%%%%%%%%%%%
%		Caratula		%
%%%%%%%%%%%%%%%%%%%%%%%%%

\input{Caratula.tex}

%%%%%%%%%%%%%%%%%%%%%
%		Indice		%
%%%%%%%%%%%%%%%%%%%%%

\newpage

%%%%%%%%%%%%%%%%%%%%%
%		Informe		%
%%%%%%%%%%%%%%%%%%%%%

\section{Reporte acerca de lo investigado hasta la actualidad}

Se ha comenzado a escribir un breve resumen a modo de marco teórico para el informe. En este se tratan temas como la modelización matemática de una imagen, discretización y cuantización de estas para su almacenado en memoria digital, distintos procesos que pueden realizarse sobre las imágenes y áreas de investigación en el procesamiento de imágenes. Se ha utilizado hasta el momento el libro \textit{R. C. Gonzalez, R. E. Woods and S. L. Eddins. Digital Image Processing Using MATLAB. Prentice Hall, 2nd ed, 2002} para obtener un conocimiento básico del tema para elegir una corriente de investigación.

En paralelo, se ha comenzado a experimentar con pequeños scripts en \textit{Python} 

\section{Resumen del trabajo}




\end{document}