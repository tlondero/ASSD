\documentclass[a4paper]{article}
\usepackage[utf8]{inputenc}
\usepackage[spanish, es-tabla, es-noshorthands]{babel}
\usepackage[table,xcdraw]{xcolor}
\usepackage[a4paper, footnotesep = 1cm, width=20cm, top=2.5cm, height=25cm, textwidth=18cm, textheight=25cm]{geometry}
%\geometry{showframe}

\usepackage{tikz}
\usepackage{amsmath}
\usepackage{amsfonts}
\usepackage{amssymb}
\usepackage{float}
\usepackage{graphicx}
\usepackage{caption}
\usepackage{subcaption}
\usepackage{multicol}
\usepackage{multirow}
\setlength{\doublerulesep}{\arrayrulewidth}
\usepackage{booktabs}
\usepackage{mathrsfs,amsmath}
\usepackage{hyperref}
\hypersetup{
    colorlinks=true,
    linkcolor=blue,
    filecolor=magenta,      
    urlcolor=blue,
    citecolor=blue,    
}

\newcommand{\quotes}[1]{``#1''}
\usepackage{array}
\newcolumntype{C}[1]{>{\centering\let\newline\\\arraybackslash\hspace{0pt}}m{#1}}
\usepackage[american]{circuitikz}
\usetikzlibrary{calc}
\usepackage{fancyhdr}
\usepackage{units} 

\graphicspath{./Imagenes}

\pagestyle{fancy}
\fancyhf{}
\lhead{22.05 ASSD}
\rhead{Mechoulam, Lambertucci, Rodriguez, Londero}
\rfoot{Página \thepage}

\begin{document}

%%%%%%%%%%%%%%%%%%%%%%%%%
%		Caratula		%
%%%%%%%%%%%%%%%%%%%%%%%%%

\begin{titlepage}
\newcommand{\HRule}{\rule{\linewidth}{0.5mm}}
\center
\mbox{\textsc{\LARGE \bfseries {Instituto Tecnológico de Buenos Aires}}}\\[1.5cm]
\textsc{\Large 22.05 Análisis de Señales y Sistemas Digitales}\\[0.5cm]


\HRule \\[0.6cm]
{ \Huge \bfseries Trabajo práctico N$^{\circ}$2}\\[0.4cm] 
\HRule \\[1.5cm]


{\large

\emph{Grupo 3}\\
\vspace{3px}

\begin{tabular}{lr} 	
\textsc{Mechoulam}, Alan  &  58438\\
\textsc{Lambertucci}, Guido Enrique  & 58009 \\
\textsc{Rodriguez Turco}, Martín Sebastian  & 56629 \\
\textsc{Londero Bonaparte}, Tomás Guillermo  & 58150 \\
\end{tabular}

\vspace{20px}

\emph{Profesores}\\
Jacoby, Daniel Andres\\
Belaustegui Goitia, Carlos F.\\
Iribarren, Rodrigo Iñaki\\
\vspace{3px}
%\textsc{} \\	

\vspace{100px}

\begin{tabular}{ll}

Presentado: & 15/05/20\\

\end{tabular}

}

\vfill

\end{titlepage}


%%%%%%%%%%%%%%%%%%%%%
%		Indice		%
%%%%%%%%%%%%%%%%%%%%%

\newpage

%%%%%%%%%%%%%%%%%%%%%
%		Informe		%
%%%%%%%%%%%%%%%%%%%%%

\section{Reporte acerca de lo investigado hasta la actualidad}

Se ha comenzado a escribir un breve resumen a modo de marco teórico para el informe. En este se tratan temas como la modelización matemática de una imagen, discretización y cuantización de estas para su almacenado en memoria digital, distintos procesos que pueden realizarse sobre las imágenes y áreas de investigación en el procesamiento de imágenes. Se ha utilizado hasta el momento el libro \textit{R. C. Gonzalez, R. E. Woods and S. L. Eddins. Digital Image Processing Using MATLAB. Prentice Hall, 2nd ed, 2002} para obtener un conocimiento básico del tema para elegir una corriente de investigación.

En paralelo, se ha comenzado a experimentar con pequeños scripts en \textit{Python} 

\section{Resumen del trabajo}




\end{document}