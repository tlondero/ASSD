\input{Header.tex}

\begin{document}

%%%%%%%%%%%%%%%%%%%%%%%%%
%		Caratula		%
%%%%%%%%%%%%%%%%%%%%%%%%%

\input{Caratula.tex}

%%%%%%%%%%%%%%%%%%%%%
%		Indice		%
%%%%%%%%%%%%%%%%%%%%%

\tableofcontents
\newpage

%%%%%%%%%%%%%%%%%%%%%
%		Informe		%
%%%%%%%%%%%%%%%%%%%%%

\section{Introducción}

En el siguiente informe se busca diseñar y simular el sistema mostrado en la Figura (\ref{fig:sist}). Para esto, se valió un oscilador que lidie con las señales de control del sistema, un filtro antialiasing y recuperador, un bloque de sample and hold y y una llave analógica.

%\section{Ejercicio 1}
%	\label{Ejercicio-1}
%	\input{../Informe/Header.tex}

\begin{document}

\subsection{Introducción}

\subsection{Máxima frecuencia de entrada sin Sample \& Hold}

De la datasheet del ADC0808, se tiene que si $V_{CC} = V_{REF+} = 5.12V$ y $V_{REF-} = 0V$, la resolución será de $20 \frac{mV}{bit}$. Si se utiliza la frecuencia de clock $f_{CLK}$ típica utilizada en la datasheet de $640kHz$, el tiempo de conversión $t_C$ máximo será de $116\mu s$. Esto implica que la entrada no deberá de tener una pendiente mayor a $\frac{20mV}{116\mu s}$ para no introducir error en la cuantización de la señal.


Si la señal de entrada se encuentra en el peor caso, es decir, con una excursión de tensión de $-0.1V + V_{REF-}$ a $5.12V + 0.1V$; esta se encuentra montada sobre un nivel de continua igual a $(5.22V - (-0.1V))/2 = 2.66V$; y esta se puede considerar senoidal gracias a la teoría desarrollada por Fourier; se tiene que la amplitud pico máxima de la senoidal podrá ser $2.66V$. Luego, asumiendo el peor caso de la pendiente de la senoidal, para un ángulo igual a cero radianes, lo que permite utilizar la aproximación paraxial, se tiene que
\\

\begin{equation}
\left. \frac{d \left( 2.66V \cdot Sin \left( 2\pi f_{in_{max}} t \right) \right)}{dt} \right|_{t=0} = 2.66V \cdot 2\pi f_{in_{max}} = \frac{20mV}{116\mu s}
\end{equation}
\\

Finalmente, se obtiene una frecuencia máxima de $f_{in_{max}} = 10.3Hz$.
\subsection{DAC}
Se utilizó el integrad DAC0800, un conversor D/A de 8 bits con salida diferencial de corriente.
Para convertir esta corriente en un nivel de tensión se utilizó el circuito propuesto por la hoja de datos que se muestra a continuación:
\begin{figure}[H]
	\centering
	\includegraphics[width=0.8\textwidth]{ImagenesEjercicio1/dacout.png}
\caption{Configuración saldia DAC.}
	\label{fig:dacout}
\end{figure}
La salida va de 0 a $V_{fs}= I_{fs}cdot R_L$.
\end{document}


\section{Filtros}
	\label{Ejercicio-2}
	\input{../Informe/Header.tex}

\begin{document}
\subsubsection{Introducción a filtros Anti Aliasing}
Un filtro \textbf{Anti Aliasing} es un filtro pasa bajos, el cual se encarga de que se cumpla el criterio de muestreo de \textbf{Nyquist}. Este postula que, para la reconstrucción exacta de una señal periódica continua en banda base, es necesaria una frecuencia de muestreo $f_s$ de un valor por lo menos 2 veces superior al ancho de banda B de la señal original. Bajo estas condiciones, para reconstruir la señal solo hace falta realizar la convolución de la muestreada con la función $sinc(B\cdot t$).

La señal original se puede expresar como:
\begin{align}
	x\left( t \right) \sim \sum_{n=-\infty}^{\infty} x\left( \frac{n}{f_s} \right) \cdot sinc \left( t-\frac{n}{f_s} \right)
\end{align}

En caso de que no se cumplan las hipótesis de Nyquist, se presenta un problema, el cual es observable utilizando el análisis de Fourier. Sea $x(t)$ una señal en tiempo continuo la cual se desea digitalizar, utilizando un muestreo ideal, se multiplica la señal por un tren periódico de deltas de período $T_s = \frac{1}{f_s}$.
\begin{align}
\delta_{T_S}(f)= \sum_{n=-\infty}^{\infty} \frac{1}{T_s} \delta(f-nf_s)
\end{align}

Así se observa la transformada de Fourier.
\begin{align}
	\mathcal{F} \{x_q(t) \} (f) =\mathcal{F} \{ \delta_{T_s} (t) \cdot x(t) \} (f)=\mathcal{F} \left\lbrace \sum_{n=-\infty}^{\infty} \delta(t-nT_s) \cdot x(nT_s)\right\rbrace (f)= X(f)\  *  \ \delta_{T_S}(f)
\end{align}

Dado que el papel de la delta en una convolución es aquel de la unidad, se nota que realizar la convolución con un tren de dicha función, es equivalente a montar el espectro de una señal sobre cada delta. El problema fundamental radica en la superposición de espectros entre una delta y la continua, como está ilustrado en la Figura (\ref{fig:alias3}). En ella se observa el caso en el cual no se cumple el teorema de Nyquist, apreciandose dicho inconveniente. En las Figuras (\ref{fig:alias4}) y (\ref{fig:aliasfin}) se termina de observar las consecuencias que esto implica.\footnote{J. G. Proakis y D. G. Manolakis, \textit{Digital signal processing}. Upper Saddle River, (Nueva Jersey): Pearson, 2014, pag 388}.

\begin{figure}[H]
	\centering
	\includegraphics[width=0.8\textwidth]{ImagenesEjercicio2/aliassignal.PNG}
\caption{Señal de entrada junto a su espectro.}
	\label{fig:aliassingal}
\end{figure}
\begin{figure}[H]
	\centering
	\includegraphics[width=0.8\textwidth]{ImagenesEjercicio2/aliasquatum.PNG}
\caption{Señal cuantizada junto a su espectro con $f_s > 2B$.}
	\label{fig:aliasquantum}
\end{figure}
\begin{figure}[H]
	\centering
	\includegraphics[width=0.8\textwidth]{ImagenesEjercicio2/alias3.PNG}
\caption{Señal cuantizada junto a su espectro con $f_s < 2B$.}
	\label{fig:alias3}
\end{figure}
\begin{figure}[H]
	\centering
	\includegraphics[width=0.8\textwidth]{ImagenesEjercicio2/alias4.PNG}
\caption{Señal cuantizada junto a su espectro resultante.}
	\label{fig:alias4}
\end{figure}
\begin{figure}[H]
	\centering
	\includegraphics[width=0.8\textwidth]{ImagenesEjercicio2/aliasfin.PNG}
\caption{Señal reconstruida junto a su espectro.}
	\label{fig:aliasfin}
\end{figure}

Otro factor importante a considerar es que la frecuencia de Nyquist está definida como $2B$ únicamente para un filtro ideal. En la realidad se debe tomar una frecuencia $f_s>2B$, dado que se utiliza un filtro real.

%\begin{figure}[H]
%	\centering
%	\includegraphics[width=0.9\textwidth]{Ejercicio6/Imagenes/SalidaVsVLM35.png}
%\caption{Tensión de salida Vs. Tensión del LM35.}
%	\label{fig:vout}
%\end{figure}

\subsubsection{Introducción a filtros recuperadores}
El filtro recuperador es aquel que cumple con la tarea de recuperar la señal a partir de su espectro, dado que previo al filtro, se tiene un espectro como es observado en la Figura (\ref{fig:recuperador}). El objetivo de este es filtrar los espectros ajenos a la banda base, para obtener en el dominio temporal la señal original. Es necesario remarcar la vital importancia de recuperador, como su nombre indica.
\begin{figure}[H]
	\centering
	\includegraphics[width=0.8\textwidth]{ImagenesEjercicio2/recuperador.PNG}
\caption{Espectro de la señal cuantizada. En rojo el filtro recuperador ideal.}
	\label{fig:recuperador}
\end{figure}

\subsubsection{Análisis espectral}
En esta sección se procede a analizar el tipo de señales que recibe el circuito. Es así que see especifican las siguientes señales:
{\center
$X_a = cos(2 \pi f_{in} t)$ \ \ $\sim$  \ \
$X_b$: $\frac{3}{2}$ seno \  \ $\sim$ \ \
$X_c$ : Señal triangular \\
}

Para esto se utiliza como herramienta la serie de Fourier de cada una de estas señales, la cual se define como:
\begin{equation*}
x(t) = \sum_{n=-\infty}^{\infty} X_n \cdot e^{j2\pi n f_0t}
\end{equation*}
\begin{equation*}
X_n= \int_{-\frac{T}{2}}^\frac{T}{2} x(t) \cdot e^{-j2\pi n f_0t} dt
\end{equation*}

Teniendo en cuenta que tambien se puede expresar de forma trigonométrica, se puden reescribir como:
\begin{equation*}
x(t) = \sum_{n=0}^{\infty} a_n \cdot \cos(2\pi n f_0t) + b_n  \cdot \sin(2\pi n f_0t)
\end{equation*}
\begin{equation*}
a_n = \int_{-\frac{T}{2}}^\frac{T}{2} x(t) \cdot \cos(2\pi n f_0t) dt
\end{equation*}
\begin{equation*}
b_n = \int_{-\frac{T}{2}}^\frac{T}{2} x(t) \cdot \sin(2\pi n f_0t) dt 
\end{equation*}

Realizando las cuentas para cada señal, se obtiene que:
\begin{center}
$X_a$ ya es su propio desarrollo en serie.
\end{center}
\begin{equation*}
X_b = \sum_{n=0}^{\infty} \frac{12}{\pi} \cdot \frac{1}{9-4n^2} \cdot \cos(2\pi n f_0t)
\end{equation*}
\begin{equation*}
X_c = \sum_{n=1,3,5,...}^{\infty} \frac{8 \cdot (-1)^{\frac{n-1}{2}}}{\pi^2 n^2} \cdot \sin(2\pi n f_0t)
\end{equation*}

Dado que estas ultimas 2 señales cuentan con infinitos armónicos, el criterio que se decide utilizar para saber hasta cual se debe conservar consiste en tomar todos los necesarios hasta obtener una potencia del $99\%$. Es útil recordar dicha variable de una señal se encuentra en sus coeficientes de Fourier, mediante la \textbf{igualdad de Parseval}:
\begin{align}
\frac{1}{T} \cdot \int_{-\infty}^\infty |\ x(t)\ |^2 \ =\  \sum_{n=- \infty}^{\infty} |\ X_n \ |^2
\end{align}

Para las señales $X_b$ y $X_c$ se graficó la potencia en función del armónico y como queda la señal reconstruida  luego de este filtro.
\begin{figure}[H]
	\centering
	\includegraphics[width=0.85\textwidth]{ImagenesEjercicio2/10Armonicos.PNG}
	\includegraphics[width=0.85\textwidth]{ImagenesEjercicio2/4ARMONICOS.PNG}
	\caption{Señal triangular reconstruida junto a su espectro de potencias.}
	\label{fig:pottriang}
\end{figure}
\begin{figure}[H]
	\centering
	\includegraphics[width=0.85\textwidth]{ImagenesEjercicio2/sen32signal.PNG}
	\includegraphics[width=0.85\textwidth]{ImagenesEjercicio2/sen32signalpot.PNG}
	\caption{Señal seno $\frac{3}{2}$ reconstruida junto a su espectro de potencias.}
	\label{fig:potsin}
\end{figure}

Es de interes apreciar que, para el caso de la señal triangular, tan solo con la inclusión de 2 armónicos, se obtiene una potencia superior al $98\%$. Por otro lado, se necesita incluir hasta el 3er armónico de la función  seno $\frac{3}{2}$ para obtener una potencia superior al $95\%$.

La frecuencia fundamental de estas señales es $f_0 = \frac{N}{2} = 1,5 \ kHz$, correspondiendole la del máximo armónico (es decir, el tercero) de $10.5 \ kHz$. Asimismo, es conveniente considerar la señal de AM que también es probada en el sistema, cuya máxima frecuencia es de $2,2 \cdot f_0 = 3.3 \ kHz$. Se toma $f_p= 11 \ kHz$  definiendo así la plantilla de ambos filtros:
\begin{multicols}{2}
\begin{itemize}
	\item $f_p = 11\ kHz$
	\item $f_a = 16.5 \ kHz$
	\item $A_p = 1 \ dB$
	\item $A_s = 50 \ dB$
\end{itemize}
\end{multicols}

Ya con los parámetros definidos, se procedió a elegir la aproximación a utilizar. Se descartaron las opciones de Cheby II y Cauer debido a los ceros de transmisión que estos poseen, Cheby I dado a su ripple de banda pasante, Guass y Bessel ya que su principal característica es la linealidad de la fase dejando para elegir Butterworth y Legendre. Si bien el primero tiene la mayor planicie de banda pasante, sufre de que, para cumplir plantilla, necesita un orden superior el que utiliza la segunda aproximación. Ademas, Legendre cuenta con el mayor cambio de pendiente. Por dichas razones, se decidió utilizar la aproximación de Legendre. Realizandola se obtuvo el diagrama de polos y ceros y una transferencia teórica, siendo estos los presentados acontinuación:
 \begin{figure}[H]
	\centering
	\includegraphics[width=0.65\textwidth]{ImagenesEjercicio2/polosyceros.PNG}
\caption{Diagrama de polos y ceros.}
	\label{fig:polosyceros}
\end{figure}
\begin{figure}[H]
	\centering
	\includegraphics[width=0.8\textwidth]{ImagenesEjercicio2/atenuation.PNG}
\caption{Respuesta en frecuencia teórica.}
	\label{fig:transteorica}
\end{figure}

Sintentizando el filtro con los datos mostrado, se corresponde un filtro de orden 10, con los valores indicados en la siguiente tabla:
\begin{table}[H]
\centering
\begin{tabular}{ccc}
\hline
\textbf{Etapa} & \textbf{Frecuencia de corte [kHz]} & \textbf{Q} \\ \hline
1 & 4.5 & 0.54 \\
2 & 6.2 & 0.84 \\
3 & 8.3 & 1.45 \\
4 & 10.1 & 2.82 \\
5 & 11.2 & 9.06 \\ \hline
\end{tabular}
\caption{Frecuencias de corte y Q de las etapas del filtro deseado.}
\end{table}

La topología elegida para realizar las etapas viene dada por la \textbf{Sallen Key} pasa bajos, debido a que no se alcanzan valores altos de Q, siendo el siguiente el circuito correspondiente.
\begin{figure}[H]
\centering
	\includegraphics[width=0.7\textwidth]{ImagenesEjercicio2/SK.pdf}
	\caption{Celda Sallen Key.}
	\label{fig:SK}
\end{figure}

Luego se obtuvieron los siguientes valores para los componentes:
\begin{multicols}{2}
\begin{table}[H]
\centering
\begin{tabular}{cccc}
\hline
Componente & Valor & Valor comercial & Error \\ \hline
$R_1$ & $100 \ k\Omega$ & $100 \ k\Omega$ & $0\%$ \\
$R_2$ & $100 \ k\Omega$ & $100 \ k\Omega$ & $0\%$\\
$C_1$ & $378.6pF \ pF$ & $390 \ pF + 12 \ nF$ & $0.2\%$ \\
$C_2$ & $326.8pF \ pF$ & $330 \ pF + 33 \ nF$ & $\le0.1\%$ \\ \hline
\end{tabular}
\caption{Componentes de la etapa 1.}
\end{table}

\begin{table}[H]
\centering
\begin{tabular}{cccc}
\hline
Componente & Valor & Valor comercial & Error \\ \hline
$R_1$ & $100 \ k\Omega$ & $100 \ k\Omega$ & $0\%$ \\
$R_2$ & $100 \ k\Omega$ & $100 \ k\Omega$ & $0\%$\\
$C_1$ & $552.6 \ pF$ & $560 \ pF + 39 \ nF$ & $\le0.1\%$ \\
$C_2$ & $65.8 \ pF$ & $82 \ pF + 330 \ pF$ & $0.2\%$ \\ \hline
\end{tabular}
\caption{Componentes de la etapa 3.}
\end{table}
\begin{table}[H]
\centering
\begin{tabular}{cccc}
\hline
Componente & Valor & Valor comercial & Error \\ \hline
$R_1$ & $100 \ k\Omega$ & $100 \ k\Omega$ & $0\%$ \\
$R_2$ & $100 \ k\Omega$ & $100 \ k\Omega$ & $0\%$\\
$C_1$ & $429.1 \ pF$ & $39 \ pF // 390 \ pF$ & $0.4\%$ \\
$C_2$ & $153.3 \ pF$ & $3.3 \ pF // 150 \ pF$ & $0.1\%$ \\ \hline
\end{tabular}
\caption{Componentes de la etapa 2.}
\end{table}
\begin{table}[H]
\centering
\begin{tabular}{cccc}
\hline
Componente & Valor & Valor comercial & Error \\ \hline
$R_1$ & $100 \ k\Omega$ & $100 \ k\Omega$ & $0\%$ \\
$R_2$ & $100 \ k\Omega$ & $100 \ k\Omega$ & $0\%$\\
$C_1$ & $884.9 \ pF$ & $68pF // 820pF $ & $0.3\%$ \\
$C_2$ & $27.9 \ pF$ & $33 \ pF + 180 \ pF$ & $\le0.1\%$ \\ \hline
\end{tabular}
\caption{Componentes de la etapa 4.}
\end{table}
\end{multicols}
\begin{table}[H]
\centering
\begin{tabular}{cccc}
\hline
Componente & Valor & Valor comercial & Error \\ \hline
$R_1$ & $100 \ k\Omega$ & $100 \ k\Omega$ & $0\%$ \\
$R_2$ & $100 \ k\Omega$ & $100 \ k\Omega$ & $0\%$\\
$C_1$ & $2.6 \ nF$ & $3.3 \ nF + 12 \ nF$ & $\le0.1\%$ \\
$C_2$ & $7.9 \ pF$ & $8.2 \ pF + 220 \ pF$ & $0.3\%$ \\ \hline
\end{tabular}
\caption{Componentes de la etapa 5.}
\end{table}

Para la implementación se optó por utilizar amplificadores del tipo \href{http://www.ti.com/lit/ds/symlink/tl082.pdf}{TL084} debido a que cada integrado cuenta con 4 opamps, a su elevada impedancia de entrada y a su ancho de banda. Se simuló en \textbf{LTSpice} el filtro completo, obteniendo así la respuesta en frecuencia del mismo como se ve a continuación:
 \begin{figure}[H]
	\centering
	\includegraphics[width=0.8\textwidth]{ImagenesEjercicio2/spice.PNG}
\caption{Respuesta en frecuencia simulada.}
	\label{fig:transspice}
\end{figure}
A continuación se hizo un análisis de montecarlo del filtro, obteniendo una ligera desviación respecto del filtro deseado, la cual aun asi se ajusta a la plantilla con la eventual ganancia en el sobrepico.
 \begin{figure}[H]
	\centering
	\includegraphics[width=0.8\textwidth]{ImagenesEjercicio2/montecarlo.PNG}
\caption{Montecarlo de la respuesta en frecuencia.}
	\label{fig:montecarlo}
\end{figure}
Luego, se realizó el diseño en \textbf{Altium} de la placa a realizar, obteniendo llegandose así al diseño presentado:
 \begin{figure}[H]
	\centering
	\includegraphics[width=0.6\textwidth]{ImagenesEjercicio2/altiumesq.PNG}
\caption{Esquemático Altium.}
	\label{fig:altiumesq}
\end{figure}
 \begin{figure}[H]
	\centering
	\includegraphics[width=0.6\textwidth]{ImagenesEjercicio2/altiumpcb.PNG}
\caption{PCB Altium.}
	\label{fig:altiumpcb}
\end{figure}

\end{document}

\section{Selección de llave analógica}
	\label{Ejercicio-3}
	\input{../Informe/Header.tex}

\begin{document}

Las llaves compuestas por tecnología de estado solido son pequeñas, rápidas, de fácil uso y control. Además poseen un consumo bajo comparado con compuertas tradicionales controladas electricamente.

Las compuertas digitales estan diseñadas para que transmitir y bloquear señales de niveles digitales. Por otro lado, las analógicas son diseñados para señales analógicas, si bien normalmente presentan un buen comportamiento frente a las digitales.

\subsection{\href{http://www.ti.com/lit/ds/symlink/cd4016b.pdf}{CD4016}}
\begin{multicols}{2}
\begin{itemize}
	\item $V_{OS} = 0.4 \ V \sim 13.5 \ V$
	\item Resistencia ``on-state'' $= 400 \ \Omega \sim 2 \ k\Omega$
	\item $TDH = 0.4\%$
	\item Capacidad de entrada $C_{is} = 4 \ pF$
	\item Capacidad de salida $C_{os} = 4 \ pF$
	\item Capacidad Feedthrough $C_{ios} = 0.2 \ pF$
	\item Crosstalk $= 50 \ mV$
	\item Delay de encendido/apagado $= 15 \ ns \sim 70 \ ns$
\end{itemize}
\end{multicols}

\subsection{\href{http://www.ti.com/lit/ds/symlink/cd4066b.pdf}{CD4066}, \href{http://www.ti.com/lit/ds/symlink/cd4051b.pdf}{CD4053} y \href{http://www.ti.com/lit/ds/symlink/cd4051b.pdf}{CD4051}}
\begin{multicols}{2}
\begin{itemize}
	\item $V_{OS} = 0.4 \ V \sim 13.5 \ V$
	\item Resistencia ``on-state'' $= 200 \ \Omega \sim 1.3 \ k\Omega$
	\item $TDH = 0.4\%$
	\item Capacidad de entrada $C_{is} = 8 \ pF$
	\item Capacidad de salida $C_{os} = 8 \ pF$
	\item Capacidad Feedthrough $C_{ios} = 0.5 \ pF$
	\item Crosstalk $= 50 \ mV$
	\item Delay de encendido/apagado $= 15 \ ns \sim 70 \ ns$
\end{itemize}
\end{multicols}
Estos datos varían dependiendo en $VDD$, entre $5 \ V$ y $15 \ V$.


\end{document}
	
\section{Sample and Hold}
	\label{Ejercicio-4}
	\input{../Informe/Header.tex}

\begin{document}
\definecolor{turquoise}{rgb}{0, 0.68627,0.68627}
\definecolor{caribbeangreen}{rgb}{0.0, 0.8, 0.6}
\subsection{Sample and Hold}

El modulo de \textbf{Sample and Hold}  puede ser esquematizado de la siguiente forma:

\begin{figure}[H]
	\centering
	\includegraphics[width=0.7\linewidth]{ImagenesEjercicio4/SyH}
	\caption{}
	\label{fig:syh}
\end{figure}
Su objetivo es el de muestrear la señal analógica de entrada y retener su valor por un pequeño intervalo de tiempo para que la circuiteria colocada inmediatamente después puede utilizar ese valor para digitalizarla.
Para esto, requerimos de una alta impedancia de entrada lo cual evita cualquier efecto de carga del IC sobre la fuente de la señal. Esto es modelado como un buffer a la entrada. Luego tenemos una \textbf{llave}, que puede estar implementada con una llave analogica o transistores MOSFET, la cual se encargara de cambiar del modo muestreo(sample) permitiendo que el capacitor se carga con el valor actual, y mantener(hold) ese valor cuando la misma este abierta. Finalmente tenemos un buffer adicional a la salida para prevenir la descarga del capacitor y así ofrecer mayor fidelidad. 


 Sin embargo, es necesario adaptarlo a las condiciones en las que se lo va a utilizar. Para esto, se incluyen pines de corrección de tensión de offset, selección tasa de muestreo (para controlar la "llave") y finalmente otro muy importante para escoger el capacitor, $C_{hold}$ más apropiado.
 Es interesante mencionar la inclusión de un pin \textbf{LOGIC REFERENCE} el cual nos brinda mayor flexibilidad al momento de tener que elegir una señal de muestreo\footnote{Para saber más sobre el funcionamiento y puesta a punto del LF398 revisar el apéndice A}.
 \subsubsection{Circuito de corrección de tensión offset}
 Para poder obtener una digitalización de alta fidelidad es necesario calibrar el IC. En el caso de la digitalización de señales las tensiones de offset pueden provocar interpretaciones erróneas de los verdaderos valores de la señal. De hecho se recomienda que la tensión de offset este por debajo de la mitad del voltaje que ofrece el LSB.
 $$
 V_{OS} < \frac{FS}{2^{n+1}} 
 $$
Donde \textit{n }es la resolución del \textbf{ADC}
y \textit{FS} es el tope de escala.

El fabricante nos brinda una configuración para poder compensar la tensión de offset tanto de AC como de DC:
\begin{figure}[H]
	\centering
	\includegraphics[scale=0.6]{ImagenesEjercicio4/DCcolorized}
	\caption{En \textcolor{red}{rojo} se señala el circuito de corrección de tensión DC, en la parte inferior notamos la presencia de un circuito que corrige la señal AC no deseada}
	\label{fig:dccolorized}
\end{figure}

\subsubsection{Análisis experimental del Capacitor de Hold, $C_{h}$}
El capacitor de hold cumple la funcionalidad de retener el valor muestreado una vez obtenido. Es deseable elegir un capacitor cuyo dieléctrico ofrezca una gran resistencia para evitar la descarga indeseada del capacitor y mantener el valor obtenido.

Se simulo el muestreo de una señal sinusoidal bajo diferentes condiciones de frecuencia de oscilación y distinto valor de $C_{hold}$

En una primera instancia se utilizo una capacitor de $47nF$. 
\begin{figure}[H]
	\centering
	\includegraphics[width=\linewidth]{ImagenesEjercicio4/ChTests/Vin1_30ksamplig47nF}
	\caption{Sinusoidal 5Khz muestreada a 30KHz con un $C_{hold}=47nF$}
	\label{fig:vin130ksamplig47nf}
\end{figure}
En color \textcolor{red}{rojo} observamos el resultado de la operación de muestreado. 
Cuando la señal de muestreo, en \textcolor{blue}{azul}, supera el nivel de la tensión de lógica de referencia(\textcolor{turquoise}{turquesa}), en este caso seleccionada en 0V, la llave "se cierra" y comienza el proceso de muestreo. Esto implica que el capacitor, $C_{hold}$, puede comenzar a cargarse con el valor que tenga señal de entrada durante el tiempo que permanezca en ese modo. Cuando la señal de muestreo se encuentra por debajo de la tensión lógica de referencia se ingresa al modo \textbf{hold}, la llave "se abre" y el capacitor retiene el último valor obtenido. No obstante vemos que la el valor de tensión almacenado en el capacitor no consigue equiparar aquel de la señal a muestrear. Recordemos que entre mayor sea el valor de la capacitancia más tiempo tardara en cargarse, lo cual afecta el desempeño en el modo de muestreo pero mejora significativamente la persistencia durante el tiempo de \textbf{hold} ya que por el contrario tarde más en cargarse.

En este caso se hace evidente que el capacitor no consigue cargarse lo suficientemente rápido.

\begin{figure}[H]
	\centering
	\includegraphics[]{ImagenesEjercicio4/ChTests/Vin1_30ksamplig47nFZoom}
	\caption{Vista en detalle del proceso}
	\label{fig:vin130ksamplig47nfzoom}
\end{figure}

Para poder exhibir el caso opuesto se utilizo un capacitor $C_{hold}$ de $120pF$. Este capacitor posee una capacitancia aproximadamente 400 veces más pequeña que la utilizada anteriormente.
En este caso la señal de salida representa más fielmente a la original. Una primera observación nos deja observar que los valores de tensión obtenidos durante el muestreo  

\begin{figure}[H]
	\centering
	\includegraphics[width=\linewidth]{ImagenesEjercicio4/ChTests/Vin1_30ksamplig120pF}
	\caption{}
	\label{fig:vin130ksamplig120pf}
\end{figure}


\begin{figure}[H]
	\centering
	\includegraphics[]{ImagenesEjercicio4/ChTests/Vin1_30ksamplig120pFZoom2}
	\caption{}
	\label{fig:vin130ksamplig120pfzoom}
\end{figure}


Analizando la figura \ref{fig:vin130ksamplig120pfzoom} notamos que el capacitor efectivamente consigue cargarse lo suficientemente rápido como para poder seguirle el paso a la señal de entrada. 


Ahora veremos lo que sucede cuando empleamos una señal más rápida.
La señal senoidal ahora posee una frecuencia de oscilación de $20*3*12 KHz = 720Khz$. Para poder obtener imágenes y apreciar los cambios en los capacitores elegidos es necesarios muestrear a la señal por encima de la mínima permitida por el criterio de Nyquist. Sin embargo, si consideramos ese caso obtenemos  $F_{nyqusit}=1.44Mhz$.
Para la simulación de la  figura \ref{fig:vin21440ksamplign120pf} se utilizo el capacitor de $120pF$ que había obtenido buenos resultados previamente. Pese a ello una imagen poco anticipada aparece al finalizar la simulación. La salida parecería ser nula. Aun si esta no fuese nula no sirve a nuestros propósitos. 
\begin{figure}[H]
	\centering
	\includegraphics[width=\linewidth]{ImagenesEjercicio4/ChTests/Vin2_1440kSamplign120pF}
	\caption{}
	\label{fig:vin21440ksamplign120pf}
\end{figure}

Entonces cabe preguntarse que ha sucedido aquí. 
En primer lugar podemos relacionar lo ocurrido aquí con lo sucedido cuando se utilizo un capacitor demasiado grande que no conseguía cargarse en la ventana de adquisición provista y por lo tanto exhibía una amplitud menor.

\begin{figure}[H]
	\centering
	\includegraphics[scale=0.6]{ImagenesEjercicio4/HoldCapAcqTime}
	\caption{}
	\label{fig:holdcapacqtime}
\end{figure}

Si elegimos muestrear a la frecuencia de Nyquist entonces le estamos dando al $C_{hold}$ un tiempo de $T_{acq}=\frac{1}{2*F_{nyquist}}$
$$T_{acq}\approx 700nS = 0.7\mu S$$
Por lo tanto, si hacemos referencia a las curvas de la figura \ref{fig:holdcapacqtime} vemos que nuestro requerimiento esta por fuera de las capacidades del \textbf{LF398}. En conclusión, necesitaríamos capacitancias extremadamente pequeñas. Lo cual tiene como contrapartida que no podrán mantener el valor almacenado durante mucho tiempo.






  


\end{document}
	

\section{Entorno de simulación}
	\label{Ejercicio-5}
	%\input{../Informe/Header.tex}

%\begin{document}

La GUI fue desarrollada con la herramiente GNU Radio, la cual es similar a simulink de matlab, permitiendo realizar simulaciones de signal prossesing en tiempo real, este fue desarrollado en C++ y python.
Siendo (\ref{fig:gnu}) la interfaz del programa,
 \begin{figure}[H]
	\centering
	\includegraphics[width=0.9\textwidth]{ImagenesEjercicio5/gnuradio.PNG}
\caption{Interfaz GNURadio.}
	\label{fig:gnu}
\end{figure}
Para correr el programa lo primero es descargar el software desde la \href{http://www.gcndevelopment.com/gnuradio/downloads.htm}{página} la versión v3.7.13.5/v1.6, luego debe abrirse la aplicación "GNURadio companion", la cual abre el entorno de simulación, luego se le debe dar al simbolo de play para comenzar.Se le debe abrir la siguiente interfaz.
 \begin{figure}[H]
	\centering
	\includegraphics[width=0.9\textwidth]{ImagenesEjercicio5/gui.PNG}
\caption{GUI.}
	\label{fig:GUI}
\end{figure}
\footnote{A veces no se refresca bien la señal triangular, para que se actualize cambie la amplitud.}
%\end{document}
	
\section{Sub-Muestreo}
	\label{Ejercicio-8}
	\input{../Informe/Header.tex}

\begin{document}
\subsection{FFT}
Se implementó la FFT utilizando el algoritmo de Cooley-Tukey de manera recursiva.
Se probó con diversas entradas reales aleatorias, de tamaño 4096, con una media temporal de 40$\mu$s.
\subsection{Programa Principal}
La GUI implementada es la siguiente:
\begin{figure}[H]
	\centering
	\includegraphics[width=0.8\textwidth]{ImagenesEjercicio8/GUI.PNG}
\caption{GUI Implementada.}
	\label{fig:gui}
\end{figure}
Permite agregar midis de cualquier duración, sintetizar cada track con el instrumento deseado, al igual que escuchar un preview del mismo, funcionalidad de pantalla completa, una sección de ayuda al usuari; Ademas permite la mezcla tracks, agregar efectos tanto a los tracks como al proyecto entero. Realizar espectrogramas de los wavs generados, pudiendo elegir el tipo de ventana, la cantidad de puntos de la FFT y el overlap. Ademas permite manejar los parametros de algunos instrumentos y efectos.\\
El front-end del programa se implemento en WxWidgets. El programa fue desarrollado en C++. \\

\end{document}
	
\end{document}