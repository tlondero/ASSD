\documentclass[a4paper]{article}
\usepackage[utf8]{inputenc}
\usepackage[spanish, es-tabla, es-noshorthands]{babel}
\usepackage[table,xcdraw]{xcolor}
\usepackage[a4paper, footnotesep = 1cm, width=20cm, top=2.5cm, height=25cm, textwidth=18cm, textheight=25cm]{geometry}
%\geometry{showframe}

\usepackage{tikz}
\usepackage{amsmath}
\usepackage{amsfonts}
\usepackage{amssymb}
\usepackage{float}
\usepackage{graphicx}
\usepackage{caption}
\usepackage{subcaption}
\usepackage{multicol}
\usepackage{multirow}
\setlength{\doublerulesep}{\arrayrulewidth}
\usepackage{booktabs}
\usepackage{mathrsfs,amsmath}
\usepackage{hyperref}
\hypersetup{
    colorlinks=true,
    linkcolor=blue,
    filecolor=magenta,      
    urlcolor=blue,
    citecolor=blue,    
}

\newcommand{\quotes}[1]{``#1''}
\usepackage{array}
\newcolumntype{C}[1]{>{\centering\let\newline\\\arraybackslash\hspace{0pt}}m{#1}}
\usepackage[american]{circuitikz}
\usetikzlibrary{calc}
\usepackage{fancyhdr}
\usepackage{units} 

\graphicspath{./Imagenes}

\pagestyle{fancy}
\fancyhf{}
\lhead{22.05 ASSD}
\rhead{Mechoulam, Lambertucci, Rodriguez, Londero}
\rfoot{Página \thepage}

\begin{document}
\section{Simulaciones Básicas}

\subsubsection{Simulaciones con Python}
Se utilizó el framework de \textit{GNURadio} para programar cada módulo del sistema encerrado en una interfaz gráfica. Esta interfaz tiene la posibilidad de visualizar al mismo tiempo tanto la señal en tiempo como su espectro en cada nodo del sistema. Se puede elegir entre señales sinusoidales, triangulares, 3/2 seno o moduladas AM como entrada.
\subsubsection{Simulaciones con LTSPICEXVII}

\end{document}