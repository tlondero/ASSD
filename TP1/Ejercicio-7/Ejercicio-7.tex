En este punto se analizará la utilización del remuestreo en una señal de AM descrita por la siguiente ecuación:
\begin{equation}
X_c = A_{MAX} \cdot \left[ \frac{1}{2} \cdot \cos (2 \pi (1.8 f_{in}) t) +\cos (2 \pi (2 f_{in}) t)  + \frac{1}{2} \cdot \cos (2 \pi (2.2 f_{in}) t) \right]
\end{equation}
Lo que se busca con el remuestreo es emular un muestreo ideal, esto se lleva a cabo al utilizar en conjunto el Sample and Hold y la llave analógica, de tal manera que la llave se encuentre cerrada mientras que  se encuentra en Hold y abierta durante el Sample. Resulta simple notar, que es equivalente a multiplicar la señal por un tren de deltas y luego convolucionar la señal con un pulso.
	
\begin{figure}[H]
	\centering
	\includegraphics[width=1\textwidth]{ImagenesEjercicio7/t1.PNG}
\caption{Señal senoidal en sus diversas etapas de muestreo.}
	\label{fig:muestreo}
\end{figure}
Luego consideraremos que las señales, dado que $f_{in} = 1.5kHz$ , tendrán una frecuencia $f_p=3kHz$ y $f_m=300Hz$ con un m=1, esto define que la máxima frecuencia del sistema será de 3.3kHz, por Nyquist la frecuencia de muestreo debe ser por lo menos de 6.6kHz, se decidió utilizar una $f_s = 15kHz$. lo cual implica un período de 34$\mu$s, es de interés saber que el tiempo de adquisición  del SH es típicamente de \href{http://www.ti.com/lit/ds/symlink/lf398-n-mil.pdf}{6$\mu$s}, por seguridad se dejara un margen de error para el sample, dejando para este el $12\mu s$, a partir de aquí se puede determinar que debe estar en Sample un 35$\%$ y por lo tanto la llave analógica debe tener un Duty cycle del 65$\%$, siendo conscientes de que a mayor Duty cycle mayor será la potencia recuperada de la señal.
\\
Finalmente se grafica la tensión en cada uno de los nodos del sistema dada la entrada AM $X_c$ tanto en una simulación como con la GUI.
\begin{figure}[H]
	\centering
	\includegraphics[width=1\textwidth]{ImagenesEjercicio7/input.PNG}
\caption{Señal de entrada AM.}
	\label{fig:input}
\end{figure}
\begin{figure}[H]
	\centering
	\includegraphics[width=1\textwidth]{ImagenesEjercicio7/sinput.PNG}
\caption{Señal de entrada AM spice.}
	\label{fig:sinput}
\end{figure}

\begin{figure}[H]
	\centering
	\includegraphics[width=1\textwidth]{ImagenesEjercicio7/alias.PNG}
\caption{Señal luego del filtro anti alias.}
	\label{fig:alias}
\end{figure}


\begin{figure}[H]
	\centering
	\includegraphics[width=1\textwidth]{ImagenesEjercicio7/salias.PNG}
\caption{Señal luego del filtro anti alias spice.}
	\label{fig:salias}
\end{figure}

\begin{figure}[H]
	\centering
	\includegraphics[width=1\textwidth]{ImagenesEjercicio7/sh.PNG}
\caption{Señal luego del Sample and Hold.}
	\label{fig:sh}
\end{figure}

\begin{figure}[H]
	\centering
	\includegraphics[width=1\textwidth]{ImagenesEjercicio7/ssh.PNG}
\caption{Señal luego del Sample and Hold spice.}
	\label{fig:ssh}
\end{figure}

\begin{figure}[H]
	\centering
	\includegraphics[width=1\textwidth]{ImagenesEjercicio7/analog.PNG}
\caption{Señal luego de la llave analógica.}
	\label{fig:analog}
\end{figure}

\begin{figure}[H]
	\centering
	\includegraphics[width=1\textwidth]{ImagenesEjercicio7/sanalog.PNG}
\caption{Señal luego de la llave analógica spice.}
	\label{fig:sanalog}
\end{figure}

\begin{figure}[H]
	\centering
	\includegraphics[width=1\textwidth]{ImagenesEjercicio7/recovery.PNG}
\caption{Señal luego del filtro recuperador.}
	\label{fig:recovery}
\end{figure}

\begin{figure}[H]
	\centering
	\includegraphics[width=1\textwidth]{ImagenesEjercicio7/srecovery.PNG}
\caption{Señal luego del filtro recuperador spice.}
	\label{fig:srecovery}
\end{figure}
Es apreciable, que el espectro de cada señal corresponde con el de una señal modulada en AM para las figuras (\ref{fig:input}) , (\ref{fig:alias}) y (\ref{fig:recovery}) mientras que se ve en las figuras (\ref{fig:sh}) y (\ref{fig:analog}) las replicas del espectro original en el resto del espectro, una observación notable, es que a simple vista los espectros entre el spice y el calculado parecen diferir, es que uno se encuentra en escala logarítmica y el otro lineal.