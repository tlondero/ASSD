\input{../Informe/Header.tex}

\begin{document}

\subsection{Introducción}

En el siguiente informe se busca diseñar y simular el sistema mostrado en la Figura (\ref{fig:sist}). Para esto, se valió un oscilador que lidie con las señales de control del sistema, un filtro antialiasing y recuperador, un bloque de sample and hold y y una llave analógica.

\subsection{Oscilador}
Para realizar el muestreo y las subsiguientes mediciones se requiere diseñar un oscilador con frecuencia y duty cycle variable. El diseño elegido es el siguiente:

\begin{figure}[H]
	\centering
		\begin{circuitikz}
			\draw
			node[dipchip](555){555}
			
			(555.pin 1) to[short] ++ (-1.3,0)
				to[short] ++ (0, -0.2)
				node[tlground]{}
				
			(555.pin 8) to [short] ++ (1,0)
				to[short] ++ (0, 0.95)
				to[short] ++ (2, 0)
				node[ocirc, label=east:$V_{in}$]{}
			
			(555.pin 7) to [short] ++ (3,0)
				node[ocirc, label=east:$V_{out}$]{}
				++ (-1, 0)
				to[R=$R_1$, *-*] ++ (0, 1.5)
				++ (-1, 0)
				to[short, *-] ++ (-6, 0)
				to[R=$R_2$, -*] ++ (0, -2.06)
				to[R=$R_3$] ++ (0, -2)
				node[ground]{}
				++(0, 2) to[pR=$RP_{freq}$, mirror, name=fpot] ++ (2.757,0)
				(fpot.wiper) to[short] ++ (0,-1.5)
					to[short] ++ (0.8, 0)
					++ (0.55, 0.646)
					to[pR=$RP_{DT}$, mirror] ++ (0, -1.3)
					to[short] ++ (0.75, 0)
					++(-1,0) ++ (0.24, 1.3)
					to[short] ++ (0.75,0)
					to[D=$D_1$] ++ (1, 0) to[short] ++ (0.5,0)
					++(-0.5,-1.3) to[D=$D_2$] ++ (-1,0)
					++(1,0) to[short] ++ (0.5, 0)
					to[short, -*] ++ (0,0.65) to[short] ++ (0,0.65)
					++ (0, -0.65) to[short, -*] ++ (1.75,0)
					to[C=$C_2$] ++ (0,-1.5) node[ground]{}
					++(0,1.5) |- (555.pin 6)
			(555.pin 5) to[short]++(2.5,0)
				to[C=$C_1$] ++ (0, -1) node[ground]{}
			
			(555.pin 4) to[short] ++ (-0.5, 0)
				to[short, -*] ++ (0, 2.615)
			
			(555.pin 2) to[short] ++ (0, 1.115)
				to[short] ++ (2.70, 0)
				to[short, -*] ++ (0, -1.675)		
			
			;
		\end{circuitikz}
	\caption{Oscilador con ajuste de frecuencia y duty cycle independientes.}
	\label{fig:osc}

\end{figure}

Este permite, con los valores mostrados más adelante, variar la frecuencia entre $9.66 \ kHz$, levemente menor a la frecuencia de corte del filtro anti-alias, y $25 \ kHz$, logrando traspasar a la frecuencia de Nyquist en un $25\%$. Además, este circuito permite configurar el duty cycle de la señal desde un $1\%$ a $99\%$ con máxima frecuencia y desde un $5\%$ a $95\%$ con mínima frecuencia. Existe, como se puede ver, una pequeña interacción entre el ajuste de frecuencia y duty cycle, lo que genera que los límites del duty cycle se achiquen al disminuir la frecuencia. A fines prácticos, se la consideró insignificante dado que los límites mínimos se cumplen. De esta forma, los valores tomados se detallan a continuación:
\begin{table}[H]
\centering
\begin{tabular}{cc}
\hline
Componente & Valor \\ \hline
$R_1$ & $2.2 \ k\Omega$ \\
$R_2$ & $10 \ k\Omega$ \\
$R_3$ & $10 \ k\Omega$  \\
$RP_{freq}$ & $4 \ k\Omega$  \\
$RP_{DT}$ & $45 \ k\Omega$  \\
$C_1$ & $10 \ nF$  \\
$C_2$ & $1 \ nF$\\ \hline
\end{tabular}
\caption{Componentes del oscilador.}
\end{table}

Una peculiaridad de esta configuración circuital del 555 es que la salida se encuentra tomada en el pin de descarga del integrado. Esta configuración funciona dado que el dicho pin y el pin de salida del integrado se encuentran en contra-fase. Esto permite realizar la carga y descarga del capacitor $C_1$ mediante la salida del integrado. Además, las resistencias $R_2$ y $R_3$ aumentan el rango de variabilidad de frecuencias y duty cycle del oscilador. Como el pin de descarga es de tipo open collector, se debe atar esta salida a la tensión de alimentación mediante una resistencia de pull up, en este caso $R_1$.

Los resultados del oscilador, con una alimentación de $5 \ V$ se muestran a continuación:
\begin{figure}
\centering
\begin{subfigure}[b]{.49\linewidth}
\includegraphics[width=\linewidth]{/ImagenesEjercicio1/DT50FMAX.png}
\caption{Onda simétrica con máxima frecuencia.}
\end{subfigure}
\begin{subfigure}[b]{.49\linewidth}
\includegraphics[width=\linewidth]{/ImagenesEjercicio1/DT50FMIN.png}
\caption{Onda simétrica con mínima frecuencia.}
\end{subfigure}

\begin{subfigure}[b]{.49\linewidth}
\includegraphics[width=\linewidth]{/ImagenesEjercicio1/DTMAXFMAX.png}
\caption{Máximo duty cycle con máxima frecuencia.}
\end{subfigure}
\begin{subfigure}[b]{.49\linewidth}
\includegraphics[width=\linewidth]{/ImagenesEjercicio1/DTMAXFMIN.png}
\caption{Máximo duty cycle con mínima frecuencia.}
\end{subfigure}

\begin{subfigure}[b]{.49\linewidth}
\includegraphics[width=\linewidth]{/ImagenesEjercicio1/DTMINFMAX.png}
\caption{Mínimo duty cycle con máxima frecuencia.}
\end{subfigure}
\begin{subfigure}[b]{.49\linewidth}
\includegraphics[width=\linewidth]{/ImagenesEjercicio1/DTMINFMIN.png}
\caption{Mínimo duty cycle con máxima frecuencia.}
\end{subfigure}

\end{figure}

\end{document}