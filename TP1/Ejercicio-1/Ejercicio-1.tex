\documentclass[a4paper]{article}
\usepackage[utf8]{inputenc}
\usepackage[spanish, es-tabla, es-noshorthands]{babel}
\usepackage[table,xcdraw]{xcolor}
\usepackage[a4paper, footnotesep = 1cm, width=20cm, top=2.5cm, height=25cm, textwidth=18cm, textheight=25cm]{geometry}
%\geometry{showframe}

\usepackage{tikz}
\usepackage{amsmath}
\usepackage{amsfonts}
\usepackage{amssymb}
\usepackage{float}
\usepackage{graphicx}
\usepackage{caption}
\usepackage{subcaption}
\usepackage{multicol}
\usepackage{multirow}
\setlength{\doublerulesep}{\arrayrulewidth}
\usepackage{booktabs}
\usepackage{mathrsfs,amsmath}
\usepackage{hyperref}
\hypersetup{
    colorlinks=true,
    linkcolor=blue,
    filecolor=magenta,      
    urlcolor=blue,
    citecolor=blue,    
}

\newcommand{\quotes}[1]{``#1''}
\usepackage{array}
\newcolumntype{C}[1]{>{\centering\let\newline\\\arraybackslash\hspace{0pt}}m{#1}}
\usepackage[american]{circuitikz}
\usetikzlibrary{calc}
\usepackage{fancyhdr}
\usepackage{units} 

\graphicspath{./Imagenes}

\pagestyle{fancy}
\fancyhf{}
\lhead{22.05 ASSD}
\rhead{Mechoulam, Lambertucci, Rodriguez, Londero}
\rfoot{Página \thepage}

\begin{document}

\subsection{Introducción}

\subsection{Oscilador}

Para realizar el muestreo y las subsiguientes mediciones se requiere diseñar un oscilador con frecuencia y duty cycle variable. El diseño elegido es el siguiente:

\begin{figure}[H]

	\centering
		\begin{circuitikz}
			\draw
			node[dipchip](555){555}
			
			(555.pin 1) to[short] ++ (-1.3,0)
				to[short] ++ (0, -0.2)
				node[tlground]{}
				
			(555.pin 8) to [short] ++ (1,0)
				to[short] ++ (0, 0.95)
				to[short] ++ (2, 0)
				node[ocirc, label=east:$V_{in}$]{}
			
			(555.pin 7) to [short] ++ (3,0)
				node[ocirc, label=east:$V_{out}$]{}
				++ (-1, 0)
				to[R=$R_1$, *-*] ++ (0, 1.5)
				++ (-1, 0)
				to[short, *-] ++ (-6, 0)
				to[R=$R_2$, -*] ++ (0, -2.06)
				to[R=$R_3$] ++ (0, -2)
				node[ground]{}
				++(0, 2) to[pR=$RP_{freq}$, mirror, name=fpot] ++ (2.757,0)
				(fpot.wiper) to[short] ++ (0,-1.5)
					to[short] ++ (0.8, 0)
					++ (0.55, 0.646)
					to[pR=$RP_{DT}$, mirror] ++ (0, -1.3)
					to[short] ++ (0.75, 0)
					++(-1,0) ++ (0.24, 1.3)
					to[short] ++ (0.75,0)
					to[D=$D_1$] ++ (1, 0) to[short] ++ (0.5,0)
					++(-0.5,-1.3) to[D=$D_2$] ++ (-1,0)
					++(1,0) to[short] ++ (0.5, 0)
					to[short, -*] ++ (0,0.65) to[short] ++ (0,0.65)
					++ (0, -0.65) to[short, -*] ++ (1.75,0)
					to[C=$C_2$] ++ (0,-1.5) node[ground]{}
					++(0,1.5) |- (555.pin 6)
			(555.pin 5) to[short]++(2.5,0)
				to[C=$C_1$] ++ (0, -1) node[ground]{}
			
			(555.pin 4) to[short] ++ (-0.5, 0)
				to[short, -*] ++ (0, 2.615)
			
			(555.pin 2) to[short] ++ (0, 1.115)
				to[short] ++ (2.70, 0)
				to[short, -*] ++ (0, -1.675)		
			
			;
		\end{circuitikz}
	\caption{Oscilador con ajuste de frecuencia y duty cycle independientes.}
	\label{fig:osc}

\end{figure}

Este permite, con los valores tomados mostrados a continuación, variar la frecuencia entre $~5kHz$, la frecuencia de corte de nuestro filtro anti-alias, y $15kHz$, logrando llegar hasta una frecuencia igual a tres veces la frecuencia de Nyquist. Además, este circuito permite configurar el duty cycle de la señal entre $~3\%$ y $~97\%$. Existe sin embargo una muy pequeña interacción entre el ajuste de frecuencia y duty cycle, pero a fines prácticos se la consideró insignificante.

Los valores tomados se detallan a continuación:

\begin{table}[H]
\centering
\begin{tabular}{cc}
\hline
Componente & Valor \\ \hline
$R_1$ & $10 \ k\Omega$ \\
$R_2$ & $10 \ k\Omega$ \\
$R_3$ & $2.2 \ k\Omega$  \\
$RP_{freq}$ & $3 \ k\Omega$  \\
$RP_{DT}$ & $30 \ k\Omega$  \\
$C_1$ & $10 \ nF$  \\
$C_2$ & $2.2 \ nF$\\ \hline
\end{tabular}
\caption{Componentes del oscilador.}
\end{table}

\end{document}